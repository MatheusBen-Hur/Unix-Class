% Copyright 2004 by Till Tantau <tantau@users.sourceforge.net>.
%
% In principle, this file can be redistributed and/or modified under
% the terms of the GNU Public License, version 2.
%
% However, this file is supposed to be a template to be modified
% for your own needs. For this reason, if you use this file as a
% template and not specifically distribute it as part of a another
% package/program, I grant the extra permission to freely copy and
% modify this file as you see fit and even to delete this copyright
% notice. 

\documentclass{beamer}
\usepackage[utf8]{inputenc}
\usepackage[brazilian]{babel}
\usepackage{color}

% There are many different themes available for Beamer. A comprehensive
% list with examples is given here:
% http://deic.uab.es/~iblanes/beamer_gallery/index_by_theme.html
% You can uncomment the themes below if you would like to use a different
% one:
%\usetheme{AnnArbor}
%\usetheme{Antibes}
%\usetheme{Bergen}
%\usetheme{Berkeley}
%\usetheme{Berlin}
%\usetheme{Boadilla}
%\usetheme{boxes}
\usetheme{CambridgeUS}
%\usetheme{Copenhagen}
%\usetheme{Darmstadt}
%\usetheme{default}
%\usetheme{Frankfurt}
%\usetheme{Goettingen}
%\usetheme{Hannover}
%\usetheme{Ilmenau}
%\usetheme{JuanLesPins}
%\usetheme{Luebeck}
%\usetheme{Madrid}
%\usetheme{Malmoe}
%\usetheme{Marburg}
%\usetheme{Montpellier}
%\usetheme{PaloAlto}
%\usetheme{Pittsburgh}
%\usetheme{Rochester}
%\usetheme{Singapore}
%\usetheme{Szeged}
%\usetheme{Warsaw}

\title{Aula 5 - Unix}

% A subtitle is optional and this may be deleted
\subtitle{Curso de Unix}

\author{PET Computa\c{c}ão}
% - Give the names in the same order as the appear in the paper.
% - Use the \inst{?} command only if the authors have different
%   affiliation.

\institute[UFSC] % (optional, but mostly needed)
{
%
  Departamento de Informática e Estatística\\
  Universidade de Santa Catarina}
% - Use the \inst command only if there are several affiliations.
% - Keep it simple, no one is interested in your street address.

\date{PET Computa\c{c}ão, 2015}
% - Either use conference name or its abbreviation.
% - Not really informative to the audience, more for people (including
%   yourself) who are reading the slides online

\subject{Curso de Unix}
% This is only inserted into the PDF information catalog. Can be left
% out. 

% If you have a file called "university-logo-filename.xxx", where xxx
% is a graphic format that can be processed by latex or pdflatex,
% resp., then you can add a logo as follows:

% \pgfdeclareimage[height=0.5cm]{university-logo}{university-logo-filename}
% \logo{\pgfuseimage{university-logo}}

% Delete this, if you do not want the table of contents to pop up at
% the beginning of each subsection:
\AtBeginSubsection[]
{
  \begin{frame}<beamer>{Sumário}
    \tableofcontents[currentsection,currentsubsection]
  \end{frame}
}

% Let's get started
\begin{document}

\begin{frame}
  \titlepage
\end{frame}

\begin{frame}{Sumário}
  \tableofcontents
  % You might wish to add the option [pausesections]
\end{frame}

% Section and subsections will appear in the presentation overview
% and table of contents.
\section{Seguran\c{c}a do sistema de arquivos}

\subsection{Acessos}

\begin{frame}{Seguran\c{c}a do sistema de arquivos}{Acessos}
   \begin{block}{}
\textcolor{black}{-rwxrw-r-\--} \textcolor{brown}{1} \textcolor{blue}{matheusb} \textcolor{orange}{pet\_computacao} \textcolor{green}{2450} \textcolor{purple}{Sept29 11:52} \textcolor{red}{file1}
\end{block}
   \begin{itemize}
   \item {\textcolor{black}{Dez dígitos que significam os acessos que cada usuário possui neste arquivo ou diretório.}}
   \item{\textcolor{brown}{Número de \textit{hard links} deste arquivo ou diretório, \textit{hard links} são essêncialmente nomes atribuídos a arquivos ou diretórios.}}
   \item{\textcolor{blue}{Usuário dono do arquivo ou diretório}}
  \item{\textcolor{orange}{Grupo dono do arquivo ou diretório}}
  \item{\textcolor{green}{Tamanho do arquivo ou diretório}}
  \item{\textcolor{purple}{Data de cria\c{c}ão do arquivo ou diretóiro}}
  \item{\textcolor{red}{Nome do arquivo ou diretório}}
  \end{itemize}
\end{frame}

\begin{frame}{Seguran\c{c}a do sistema de arquivos}{Acessos}
\begin{block}{}
\textcolor{black}{-rwxrw-r-\--} \textcolor{brown}{1} \textcolor{blue}{matheusb} \textcolor{orange}{pet\_computacao} \textcolor{green}{2450} \textcolor{purple}{Sept29 11:52} \textcolor{red}{file1}
\end{block}
\begin{itemize}
\item {Os primeiros dez dígitos estão relacionados as permissões de acesso ao arquivo ou diretório, o primeiro dígito pode ser \textbf{d} ou \textbf{-}, quando este dígito é \textbf{d} significa que o que foi listado é um diretório, quando é\textbf{\ -} significa que o que foi listado é um arquivo.\\ }
\textbf{Podemos dividir os nove dígitos que sobraram em três grupos:}\\
\item{Os primeiros três dígitos revelam as permissões do dono do que foi listado.}
\item{Os dígitos do meio dizem as permissões do grupo dono do que foi listado.}
\item{Os três últimos dígitos mostram as permissões dos outros usuários em rela\c{c}ão ao que foi listado.}
\end{itemize}
\end{frame}

\subsection{Modificando acessos}
\begin{frame}{Seguran\c{c}a do sistema de arquivos}{Modificando acessos}
  \begin{itemize}
  \item {O comando \textbf{chmod} permite modificar as permissões de acessos aos arquivos, pode ser usado desta forma: \textbf{chmod a$+$x arquivo}, isto sinaliza que os outros usuários terão a permissão de executa-lo.}\end{itemize}
  \begin{center}
 \begin{tabular}{||c | p{9cm}||} 
 \hline
 \textbf{Flag} & \textbf{Descri\c{c}ão}\\ [0.5ex] 
 \hline\hline
 u & Usuário\\ 
 \hline
 g & Grupo\\
 \hline
 a & Todos os usuários\\
 \hline
 o & Outro\\
 \hline
 r & Ler\\
 \hline
 w & Escrever ou deletar\\
 \hline
 x& Executar\\
 \hline
 + & Adicionar permissão\\
 \hline
 - & Remover permissão\\
 \hline
\end{tabular}
\end{center}
\end{frame}

\begin{frame}{Seguran\c{c}a do sistema de arquivos}{Modificando acessos}
  \begin{center}
 \begin{tabular}{l | r} 
 rwx rwx rwx = 111 111 111 & rwx = 111 em binário = 7\\ 
 rw- rw- rw- = 110 110 110 & rw- = 110 em binário = 6\\
 rwx --- --- = 111 000 000 & r-x = 101 em binário = 5\\
 & r-- = 100 em binário = 4\\
\end{tabular}
\end{center}

  \begin{center}
 \begin{tabular}{||c | p{9cm}||} 
 \hline
 \textbf{Permssão} & \textbf{Descri\c{c}ão}\\ [0.5ex] 
 \hline\hline
 777 & Sem restrições, qualquer um pode fazer o que quer.\\ 
 \hline
 755 & O dono do arquivo pode ler, escrever e executar. Qualquer outro usuário pode ler e executar.\\
 \hline
 700 & O dono do arquivo pode ler, escrever e executar.\\
 \hline
 666 & Todos os usuários podem ler e escrever o arquivo.\\
 \hline
 644 & O dono do arquivo pode ler e escrever e os outros só podem ler.\\
 \hline
 600 & O dono pode ler e escrever.\\
 \hline
\end{tabular}
\end{center}
\end{frame}

\section{Processos e programas em execu\c{c}ão}
\subsection{Execu\c{c}ões em Background e Foreground}
\begin{frame}{Processos e programas em execu\c{c}ão}{Execu\c{c}ões em Background e Foreground}
  \begin{itemize}
  \item { Processos são programas em execu\c{c}ão neste terminal, cada programa em execu\c{c}ão tem o seu PID (process identifier) para listar os processos e seus PID's usamos o comando \textbf{ps}. Processos podem estar em execu\c{c}ão em primeiro plano (foreground), execu\c{c}ão em segundo plano (background) ou suspenso.Usualmente o shell não retorna o Unix prompt enquanto um processo está em execu\c{c}ão em primeiro plano.
    }\end{itemize}
\end{frame}

\begin{frame}{Processos e programas em execu\c{c}ão}{Rodando processos em background}
  \begin{itemize}
  \item {Para executarmos um processo em background devemos colocar o caractere \textbf{\&} ao final da chamada do processo. Ao fazer isto não precisamos esperar o final da execu\c{c}ão do processo para continuar usando o mesmo terminal.}
  \item {Quando queremos jogar para background um processo que já está em execu\c{c}ão no terminal, devemos apertar \textit{Ctrl+z} (irá suspender o processo) e depois dar o comando \textbf{bg}.}
  \end{itemize}
\end{frame}

\begin{frame}{Processos e programas em execu\c{c}ão}{Listando processos suspensos ou em background}
  \begin{itemize}
  \item {Para listar processos suspensos ou em background no terminal utilizado devemos usar o comando \textbf{jobs}}
  \item {Se quisermos executar um processo que está suspenso, devemos dar o comando \textbf{fg}\textit{ \%numero\_do\_processo} para executa-lo em foreground ou \textbf{bg}\textit{ numero\_do\_processo} para executa-lo em background.}
  \end{itemize}
\end{frame}

\begin{frame}{Processos e programas em execu\c{c}ão}{Cancelando processos em execu\c{c}ão}
  \begin{itemize}
  \item {Quando o processo está em execu\c{c}ão em primeiro plano (foreground) e queremos cancelá-lo, podemos simplesmente apertar \textit{Ctrl+c}.}
  \item {Se o processo que queremos cancelar está em segundo plano (background) devemos dar o comando \textbf{kill} \textit{ \%numero\_do\_processo}.}
  
  \item{Podemos também cancelar um processo pelo seu PID (encontrado quando utilizamos o comando \textbf{ps}), utilizando o comando \textbf{kill} \textit{ PID\_do\_processo}, caso um processo esteja resistindo ao cancelamento, podemos utilizar \textbf{kill -9} \textit{ numero\_do\_processo}.. }
  \end{itemize}
\end{frame}

\section*{Sumário dos comandos}

\begin{frame}{Sumário dos comandos}
\begin{center}
 \begin{tabular}{|| c | p{7cm}||} 
 \hline
 \textbf{Comando} & \textbf{Descri\c{c}ão}\\ [0.5ex] 
 \hline\hline
 ls -lag & Lista as permissões de acessos para cada arquivo ou diretório\\ 
 \hline
 chmod [op\c{c}ões arquivo] & Modifica as permissões do arquivo especificado\\
 \hline
 comando \& & Executa o comando em background\\
 \hline
 Ctrl+z & Suspende processos que estão executando em foreground\\
 \hline
 Ctrl+c & Cancela processos que estão executando em foreground\\
 \hline
 bg & Executa em background o processo que está em suspensão\\
 \hline
 jobs & Lista os processos em background e suspensão\\
 \hline
\end{tabular}
\end{center}
\end{frame}

\begin{frame}{Sumário dos comandos}
\begin{center}
 \begin{tabular}{|| c | p{9cm}||} 
 \hline
 \textbf{Comando} & \textbf{Descri\c{c}ão}\\ [0.5ex] 
 \hline\hline
 fg \% & Executa em foreground o processo que está em suspensão\\
 \hline
 ps & Lista os processos em execu\c{c}ão\\
 \hline
 kill \% ou kill & Cancela o processo que está executando em background pelo seu número ou PID\\
 \hline
\end{tabular}
\end{center}
\end{frame}

\end{document}



